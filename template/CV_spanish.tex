%% The MIT License (MIT)
%%
%% Copyright (c) 2015 Daniil Belyakov
%%
%% Permission is hereby granted, free of charge, to any person obtaining a copy
%% of this software and associated documentation files (the "Software"), to deal
%% in the Software without restriction, including without limitation the rights
%% to use, copy, modify, merge, publish, distribute, sublicense, and/or sell
%% copies of the Software, and to permit persons to whom the Software is
%% furnished to do so, subject to the following conditions:
%%
%% The above copyright notice and this permission notice shall be included in all
%% copies or substantial portions of the Software.
%%
%% THE SOFTWARE IS PROVIDED "AS IS", WITHOUT WARRANTY OF ANY KIND, EXPRESS OR
%% IMPLIED, INCLUDING BUT NOT LIMITED TO THE WARRANTIES OF MERCHANTABILITY,
%% FITNESS FOR A PARTICULAR PURPOSE AND NONINFRINGEMENT. IN NO EVENT SHALL THE
%% AUTHORS OR COPYRIGHT HOLDERS BE LIABLE FOR ANY CLAIM, DAMAGES OR OTHER
%% LIABILITY, WHETHER IN AN ACTION OF CONTRACT, TORT OR OTHERWISE, ARISING FROM,
%% OUT OF OR IN CONNECTION WITH THE SOFTWARE OR THE USE OR OTHER DEALINGS IN THE
%% SOFTWARE.

% The font could be set to Windows-specific Calibri by using the 'calibri' option
\documentclass[]{mcdowellcv}

% For mathematical symbols
\usepackage{amsmath}
\usepackage[utf8]{inputenc}
\usepackage[spanish]{babel}

\usepackage{libertine}
\usepackage{libertinust1math}
\usepackage[T1]{fontenc}

% Set applicant's personal data for header
\name{Héctor Pérez Muñoz}
\address{Obispo Umaña \#033 \linebreak Estación Central, Chile.}
\contacts{(+569) 7440 3617 \linebreak hector.perez.m@hotmail.com \linebreak https://hepem.cl}

\begin{document}

	% Print the header
	\makeheader
	
	% Print the content
	\begin{cvsection}{Experiencia}
		\begin{cvsubsection}{Desarrollador de \textit{Software}}{CITIAPS}{Oct 2019 -- Actualidad}
			\begin{itemize}
				\item Centro de I+D+i que integra diferentes áreas de la Ingeniería y Psicología para generar productos de innovación social, pública y empresarial.
				\item Desarrollo, mantenimiento y soporte a aplicación geoinformática del UGIT-ONEMI. Como principales tecnologías utilizadas fueron Javascript y los servicios de ArcGIS.
				\item Desarrollo y mantención de proyecto Banco de Tiempo (República y San Miguel) desarrollado por CITIAPS.
				\item Desarrollo, soporte y mantención de proyecto FIC Los Lagos. Las principales tareas fueron desplegar la aplicación Mapstore y las base de datos GeoServer y PostgreSQL.  
			\end{itemize}
		\end{cvsubsection}
		
		\begin{cvsubsection}{Desarrollador de \textit{Software}}{WEN}{Nov 2020 -- Actualidad}		
			\begin{itemize}
				\item Startup dedicada a la salud bucal de las personas. Mis principales tareas son el desarrollo del \textit{backend} para la aplicación móvil y \textit{web}.
			\end{itemize}
		\end{cvsubsection}
	\end{cvsection}
	
	\begin{cvsection}{Educación}
		\begin{cvsubsection}{Santiago, Chile}{Universidad de Santiago}{Mar 2014 -- Actualidad}
			\begin{itemize}
				\item Ingeniería de Ejecución en Computación e Informática.
			\end{itemize}
		\end{cvsubsection}
		\begin{cvsubsection}{Santiago, Chile}{Universidad de Santiago}{Sept 2019 -- Actualidad}
			\begin{itemize}
				\item Magíster en Ingeniería Informática.
			\end{itemize}
		\end{cvsubsection}
	\end{cvsection}
	
	\begin{cvsection}{Experiencia técnica}
		\begin{cvsubsection}{Proyectos}{}{}
			\begin{itemize}
				\item \textbf{Blockchain} (2020). Basado en el libro ``Learn Blockchain by Building One" por Daniel van Flymen. Presenta conceptos claves de blockchain mediante mini-proyectos desarrollados en Python.
				\item \textbf{Portafolio} (En construcción). Página personal y portafolio desarrollada en Gatsby.js (https://hepem.cl).
				\item \textbf{Btrack} (En construcción). Herramienta CLI que permite rastrear el precio de libros en diferentes páginas chilenas.
			\end{itemize}
		\end{cvsubsection}
	\end{cvsection}
	
	\begin{cvsection}{Experiencia Adicional, Participaciones y Logros}
		\begin{cvsubsection}{}{}{}	
			\begin{itemize}
				\item \textbf{Profesor de Lab. (Mar 2020 – Mar 2021):} Asignatura de Fundamentos de Computación y Programación.
				\item \textbf{Ayudante de Lab. (Sept 2019 – Mar 2020):} Asignatura de Fundamentos de Computación y Programación.
				\item \textbf{Lion's Up (Mar 2018 – Ago 2018):} Participación en programa de emprendimiento realizado por la Universidad de Santiago.
			\end{itemize}
		\end{cvsubsection}
	\end{cvsection}
	
	\begin{cvsection}{Lenguajes y Tecnologías}
		\begin{cvsubsection}{}{}{}	
			\begin{itemize}
				\item \textbf{Lenguajes: } Javascript; Python; SQL; C++; Java; R; C. 
				\item \textbf{Frameworks: } Vue.js; Node.js/Express; Keras/Tensorflow. 
				\item \textbf{Bases de datos: } PostreSQL; MongoDB.
				\item \textbf{Marcado: } HTML5; CSS3; Markdown.
				\item \textbf{Otros: } Git; Github/Gitlab; Latex. 

			\end{itemize}
		\end{cvsubsection}
	\end{cvsection}

	\begin{cvsection}{Idiomas}
		\begin{cvsubsection}{}{}{}	
			\begin{itemize}
				\item \textbf{Español:} Nativo. 
				\item \textbf{Inglés:} Competente en lectura y audición. Nivel medio en escritura y expresión.
			\end{itemize}
		\end{cvsubsection}
	\end{cvsection}
	
\end{document}
